%--------------------------------------------------------%
  % Journal Article Manuscript Template

%% Copyright 2014 Nicholas E. Reith.
  
  % This work may be distributed and/or modified under the
  % conditions of the LaTeX Project Public License, 
  % either version 1.3 of this license or (at your option) 
  % any later version.
  
  % The latest version of this license is in
  %   http://www.latex-project.org/lppl.txt
  % and version 1.3 or later is part of all distributions of LaTeX
  % version 2005/12/01 or later.
  
  % This work has the LPPL maintenance status `unmaintained'.
  
  % This work consists of the files main.tex and subsidiary files:
  % (references: biblio.bib), (sections: content.tex, coverpage.tex,
  % preamble.tex, title_authors.tex, and title_noauthors.tex), and
  % (tables: sample_longtable.tex, and sample_table.tex).
  
  % It also includes the BibTeX reference styles: 
  % asr.bst, and ajs.bst, borrowed from here:
  % http://people.ku.edu/~chkim/

%--------------------------------------------------------%


%--------------------------------------------------------%
%	This is the main .tex file, which controls and combines
%	the other sections.
%	See the "sections" folder to edit content
%--------------------------------------------------------%

\input{sections/preamble} % See preamble.tex to edit the overall layout

% Header from Page Three on: Edit below for left and right headers
\lhead{}
\rhead{Infrastructure Resilience}

%--------------------------------------------------------%
%	BEGIN DOCUMENT
%--------------------------------------------------------%

\begin{document}

%TC:ignore   %% This region is ignored for the word count
% COVER PAGE

%--------------------------------------------------------%
%	COVER PAGE
%--------------------------------------------------------%

\begin{titlepage}

  \newcommand{\HRule}{\rule{\linewidth}{0.5mm}} % Defines a new command for the horizontal lines, change thickness here

  \center % Center everything on the page


%	HEADING SECTION

  %\textsc{\LARGE University Name}\\[1.5cm] % Name of your university/college
  \textsc{\Large Manuscript Submission}\\[0.5cm] % Major heading such as course name
  \textsc{\large Risk Analysis (An International Journal by Wiley)}\\[0.5cm] % Minor heading such as course title

%	TITLE SECTION

  \vspace{1.5 cm}
  \HRule \\[0.4cm]
  { \huge \bfseries A Hybrid Methodology for Infrastructure Resilience
    Measurement: the Impact of Electric Power Distribution Failures
    using the City of Austin and post-Maria Puerto Rico as Case
    Studies }\\[0.4cm] % Title of your document
  \HRule \\[1.5cm]
 
%	AUTHOR SECTION
%
%  {\em\Large\textbf Authors:}\\
%  \vspace{.5 cm}
%  Roy \textsc{Emanuel II}\\
%  Center for Technology and Systems Management\\
%  Department of Civil and Environmental Engineering\\
%  University of Maryland, College Park\\
%  \vspace{.5 cm}
%  Bilal M. \textsc{Ayyub}\\
%  Center for Technology and Systems Management\\
%  Department of Civil and Environmental Engineering\\
%  University of Maryland\\
%  \vspace{.5 cm}
%  Brian \textsc{Donohue}\\
%  Force Projection Sector\\
%  The Johns Hopkins University Applied Physics Laboratory\\
%
%	DATE SECTION

  \vspace{1.5 cm}
  {\large Submitted: \today}\\[3cm] % Date, change the \today to a set date if you want to be precise


\vfill % Fill the rest of the page with whitespace

\end{titlepage}

%-----------------------------------------------------------------

\newpage
 % Comment out to remove cover page
%TC:endignore

\thispagestyle{empty} % Removes header on page two. Only needed if there is a coverpage

% TITLE SECTION

% Comment out one of the two lines below to include or exclude authors in article title. You may want to exclude them in case of a double-blind peer review submission, since authors appear on the cover page, and reviewers should not see the authors' names in the manuscript draft.

%%--------------------------------------------------------%
%	TITLE SECTION with authors
%--------------------------------------------------------%

% Article title
  \title{\vspace{-15mm}\fontsize{21pt}{10pt}\selectfont\textbf{A
      Hybrid Methodology for Infrastructure Resilience Measurement:
      the Impact of Electric Power Distribution Failures using the
      City of Austin and post-Maria Puerto Rico as Case Studies
      \thanks{Draft manuscript. Please do not cite 
        without the authors' permission.}}} 

  \author[1]{\large First Author}
  \author[1]{\large Second Author}
  \author[2]{\large Third Author}
  \affil[1]{\normalsize Department of XYZ, The University of ABC}
  \affil[2]{\normalsize Department of ABC, The University of XYZ}
  \renewcommand\Authands{ and }

% Today's date
	\date{Submitted: \usvardate\today}

%-----------------------------------------------------------
%	TITLE SECTION - with authors
%-----------------------------------------------------------

% Article title
  \title{\vspace{-15mm}\fontsize{21pt}{10pt}\selectfont\textbf{A
      Hybrid Methodology for Infrastructure Resilience Measurement:
      the Impact of Electric Power Distribution Failures using the
      City of Austin and post-Maria Puerto Rico as Case Studies
      \thanks{Draft manuscript. Please do not cite without the
        authors' permission.}}}  

% Today's date
	\date{Submitted: \usvardate\today}

\makeatletter
\renewcommand{\@maketitle}{
\newpage
 \null
 \vskip 2em%
 \begin{center}%
  {\LARGE \@title \par}%
 \end{center}%
 \par} \makeatother


\maketitle % Insert title
% * <ba@umd.edu> 2018-04-30T14:39:04.308Z:
% 
% Change the title to match cover page:
% A Hybrid Methodology for Infrastructure Resilience Measurement: the
% Impact of Electric Power Distribution Failures using the City of Austin and post-Maria Puerto Rico as Case Studies
% 
% Should we add "Electric Power Distribution" in the title... perhaps..
% 
% Electric Power Distribution Infrastructure Resilience Measurement with Case Studies for the City of Austin and Puerto Rico 
% 
% 
% ^.
% NOTE: Comment out the lines below to remove line numbers
  % Running line numbers:
  \linenumbers
% * <ba@umd.edu> 2018-04-30T14:39:39.273Z:
%
% ^.
  \setlength\linenumbersep{15pt}
  \renewcommand\linenumberfont{\normalfont\footnotesize\sffamily\color{gray}}
  %\pagewiselinenumbers % Same, but that reset on every page:
  \modulolinenumbers[1] % Number only every line. Change for fewer.

%--------------------------------------------------------%
%	CONTENT
%--------------------------------------------------------%

% See the content.tex file for the abstract and body text
\input{sections/content}

%--------------------------------------------------------%
%	REFERENCE LIST
%--------------------------------------------------------%
\pagebreak
\singlespacing

% BIBTEX

    % NOTE: Comment out next two lines if using biblatex
    \bibliographystyle{references/ajs}
	\bibliography{Mendeley} % Insert bibliography

% BIBLATEX

	% NOTE: Uncomment next line to use biblatex
	% \printbibliography

%--------------------------------------------------------%
%	END NOTES
%--------------------------------------------------------%

	% Uncomment these three lines below if using endnotes
    % instead of footnotes
      % \pagebreak
      % \section*{End Notes}
      % \theendnotes

%--------------------------------------------------------%
%	FIGURES
%--------------------------------------------------------%

% NOTE: If your submission guidelines don't require figures
% at the end, you can comment out the two lines below,
% and embed figures in the text of content.tex instead of here

\pagebreak
\section*{Figures}


%figure 1
\begin{figure}[h]
\centerline{\includegraphics[width=6in]{figures/PuertoRicoMap.pdf}}
\caption{Map of Puerto Rico's Energy Infrastructure}
\label{f:PRMap}
\end{figure}
\pagebreak

%figure 2
\begin{figure}[h]
  \label{f:austinmap}
  \includegraphics[width=3in]{figures/austinmap.png}
  \caption{Map of Electricity Service for Travis County, Texas from \emph{http://austinenergy.com}}
\end{figure}
\pagebreak

%figure 3
\begin{figure}[h]
  \centerline{\includegraphics[width=6in]{figures/ClassIII.png}}
  \caption{Generalized Hybrid Model from Shanthikumar1983}
  \label{f:HybridModel}
\end{figure}
\pagebreak


%\begin{figure}[h]
%  \begin{center}
%    \includegraphics[width=3in]{figures/performanceTrajectories}
%  \end{center}
%  \renewcommand{\baselinestretch}{1}
%  \small\normalsize
%  \begin{quote}
%    \caption[Figure with caption indented]{Performance trajectories
%      adapted from \cite{Ayyub2014}}
%    \label{f:perfTraj}
%  \end{quote}
%\end{figure}


%figure 4
\begin{figure}[h]
  \centerline{\includegraphics[width = 6in]{figures/AllCaseStudies.png}}
  \label{f:CaseStudyMethodology}
  \caption{Data flow through hybrid model for both case studies}
\end{figure}
\pagebreak

%figure 5
\begin{figure}[h]
  \label{f:Maria}
  \includegraphics[width = 6in]{figures/maria}
  \caption{The Peak Load and Customer Services Restored Values in the
    Aftermath of Hurrican Maria in Puerto Rico}
\end{figure}
\pagebreak


%figure 6
\begin{figure}[h]
  \label{f:TS_SingleStorm}
  \includegraphics[width=6in]{figures/TimeSeriesSingle.png}
  \caption{Time-series example for a Category 4 storm}
\end{figure}
\pagebreak

%figure 7
\begin{figure}[h]
  \label{f:SingleStorm}
  \includegraphics{figures/SingleStormResilience.png}
  \caption{Infrastructure resilience values for 1000 simulation runs}
\end{figure}
\pagebreak

%figure 8
\begin{figure}[h]
  \label{f:SingleStrongestStorm}
  \includegraphics{figures/singlestormstrength.png}
  \caption{Infrastructure resilience values by hurricane category}
\end{figure}
\pagebreak
% * <ba@umd.edu> 2018-04-30T16:56:12.696Z:
% 
% Figure 6 is unclear
% What is "Strongest_Storm"
% What are the 1, 2, 3, and 4?
% 
% What are * shown? Hurricane Maria circle?
% 
% ^.

%figure 9
\begin{figure}[h]
  \label{f:exampleProfile}
  \includegraphics[width=5in]{figures/exampleProfile.png}
  \caption{An example of the figure of merit and endogenous preference
    over time for a ten year time horizon}
\end{figure}
\pagebreak

%figure 10
\begin{figure}[h]
  \label{f:10yrRes}
  \includegraphics{figures/10yrRes.png}
  \caption{Resilience values for each time horizon and infrastructure pair}
\end{figure}
\pagebreak

%

%figure 10
%\begin{figure}[h]
%  \label{f:allAsis}
%  \includegraphics[width=5in]{figures/allasisplot.png}
%  \caption{Resilience for each infrastructure for each stakeholder
%    profile} 
% * <ba@umd.edu> 2018-04-30T16:58:22.646Z:
% 
% Figure 9
% Spell out the terms/phrases on the abscissa (you may rotate the text counterclockwise if needed)
% 
% ^.
%\end{figure}

%--------------------------------------------------------%
%	TABLES
%--------------------------------------------------------%

% NOTE: If your submission guidelines don't require tables
% at the end, you can comment out the two lines below,
% and embed tables in the text of content.tex instead of here

\pagebreak
\section*{Tables}

%\input{tables/sample_longtable}

\input{tables/S-O-S.tex}

\pagebreak
\input{tables/resParam.tex}

\pagebreak
\input{tables/Storms.tex}

\pagebreak
\input{tables/StormCharacteristics.tex}

\pagebreak
\input{tables/FailParameters.tex}

\pagebreak
\input{tables/spikeNeed.tex}

\pagebreak
\input{tables/nProfiles.tex}

\pagebreak
\input{tables/Resilience.tex}

\pagebreak
\input{tables/stormdata.tex}

%\input{tables/sample_table}

%--------------------------------------------------------%
%	WORD COUNT
%--------------------------------------------------------%

% NOTE: Comment out this section, if you wish 
% to remove this Word Count page.

% NOTE: AJS wants no more than about 10,000 words total.

%\newpage

%TC:ignore  %% This region is ignored for the word count
%\section*{Word Count}

%This document contains approximately {\large \textbf{\wordcount}} before this page, including: title, authors, section headings, abstract, body text, tables, figure captions and references. It seems to ignore most \LaTeX \ code and \%comments, but it treats contractions such as "can't" as one word.
%TC:endignore

%--------------------------------------------------------%
%	END DOCUMENT
%--------------------------------------------------------%

\end{document}
