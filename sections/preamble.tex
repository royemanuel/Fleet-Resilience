% GENERAL MANUSCRIPT TEMPLATE

	% by Nicholas E. Reith

	% DATE: November 26th, 2014

%--------------------------------------------------------%
%	PREAMBLE
%--------------------------------------------------------%

% DOCUMENT CLASS
    % Change "letterpaper" to "a4" if you use a4 paper size
    \documentclass[letterpaper,12pt]{article}
  
% TITLE SECTION
	
 %Abstract
    \usepackage{abstract} % Allows abstract customization
    % Set the "Abstract" text to bold
    \renewcommand{\abstractnamefont}{\normalfont\bfseries}
    % Set the abstract itself to small italic text
    \renewcommand{\abstracttextfont}{\normalfont\small\itshape} 

 %Title
    \usepackage{titlesec} % Allows customization of titles

 %Authors
    \usepackage{authblk} % For multiple authors

 %Date
	\usepackage{datetime} % allows for including today's date
  	% These two lines creates a new date format ``Month day(th), year''
    \newdateformat{usvardate}{
  	\monthname[\THEMONTH] \ordinal{DAY}, \THEYEAR}

% HEADERS & FOOTERS

 %Footnotes
  	\usepackage[bottom]{footmisc} % Makes footnotes stick to bottom of the page
    
 %Endnotes
	% Uncomment this line if using endnotes "\endnote{}"
	% \usepackage{endnotes}
    
 %Headers from page 2 on
    \usepackage{fancyhdr}
    \pagestyle{fancy}
    \fancyheadoffset{0cm}
    \setlength{\headheight}{15pt} 
 
% DRAFT WATERMARK
	% NOTE: Comment out these two lines to remove watermark
    \usepackage[firstpage]{draftwatermark} % adds draft watermark
    \SetWatermarkLightness{0.85}

% MACROS
    % Define keywords macro command
    \providecommand{\keywords}[1]{\textbf{\textit{Keywords---}} #1}

%% LianTze 7 Dec 2016:
%% Updated how the wordcount is implemented
\newcommand\wordcount{%
  \immediate\write18{texcount -utf8 -merge -sum -incbib -dir -sub=none -brief \jobname.tex | cut -d : -f 1 > 'count.txt'}%
  \input{count.txt}\ignorespaces words%
}
  

% MATH SUPPORT
    % The amssymb package provides various useful mathematical symbols
    \usepackage{amssymb}
    % The amsthm package provides extended theorem environments
    \usepackage{amsthm}
    % The newtxmath package provides additional math symbol support
    	% in Times New Roman symbols, etc.
    \usepackage{newtxmath}

% FONTS
    \usepackage{microtype} % Slightly tweak font spacing for aesthetics
    \usepackage[utf8]{inputenc}
    \usepackage{newtxtext} % Makes default font Adobe Times New Roman
  
% LINES
	% Spacing
	\usepackage{setspace} % See \doublespacing command at the top of content.tex
    % Numbering
    \usepackage{lineno,xcolor} 	% See \linenumbers at the top of content.tex

% MARGINS
	%NOTE: All spaces in this template are in inches, because it is
    % formatted for letterpaper (8.5 x 11 inch) paper. If you use a4
    % paper, choose different sizes in millimeters or centimeters.
	\usepackage[top=1.5in, bottom=1.5in, left=1in, right=1in]{geometry}

% COMMENTS
	\usepackage[colorinlistoftodos]{todonotes} % allows margin comments
    % See examples in content.tex, and here for manual: 
    % http://www.ctan.org/pkg/todonotes
	\usepackage{soul} % allows for highlighting
    
% GRAPHICS
    \usepackage{graphicx} % More advanced figure inclusion
    \usepackage{float} % For specifying table/figure locations, i.e. [ht!]
    % \usepackage{flafter}    
    % The printlen command allows the user to print the exact text width or height.
    % This is useful, when trying to create graphics (outside of LaTeX, of course)
    % with the optimal dimensions. See here for usage: http://www.ctan.org/pkg/printlen
    \usepackage{printlen}

% TABLES
    \usepackage{longtable} % For long tables that span multiple pages
    \newcommand{\sym}[1]{\rlap{#1}}% For symbols like *** in tables
    \usepackage{tabularx} % Allows advanced table features
    \usepackage{makecell}
    \newcolumntype{L}[1]{>{\raggedright\arraybackslash}p{#1}}
    \newcolumntype{C}[1]{>{\centering\arraybackslash}p{#1}}
    \newcolumntype{R}[1]{>{\raggedleft\arraybackslash}p{#1}}
    \usepackage{relsize} % Allows precise adjustment of font size,
    	%useful for fitting tables to page width
	\usepackage{multirow}

% REFERENCES
	\usepackage{hyperref}
        %\usepackage{natbib}
    % For hyperlinks in the PDF

      % NOTE: This document uses bibtex for references
      % because both style files for Sociology Journals are
      % only available in .bst format. You can change to
      % biblatex if you prefer below.

      % Many thanks to Sociology Professor, 
      % ChangHwan Kim at the University of Kansas. 
      % for providing these .bst files here: 
      % http://people.ku.edu/~chkim
      
	% BIBTEX
    % Comment out this line if using biblatex
    \usepackage{chicago} % AJS and ASR styles rely on chicago

    % BIBLATEX
    % NOTE: Uncomment out these three lines to use biblatex
    % Be sure to put the biblio.bib file in biblatex format'
    %\usepackage{csquotes}
	%\usepackage[style=authoryear,backend=biber]{biblatex}
	%\bibliography{biblio}
    
